\documentclass{resume}
\usepackage{zh_CN-Adobefonts_external} 
\usepackage{cite}
\usepackage{hyperref}
\usepackage{fontawesome}
\begin{document}
\pagenumbering{gobble}

% custom rule color

%***"%"后面的所有内容是注释而非代码,不会输出到最后的PDF中
%***使用本模板,只需要参照输出的PDF,在本文档的相应位置做简单替换即可
%***修改之后,输出更新后的PDF,只需要点击Overleaf中的“Recompile”按钮即可
%**********************************姓名********************************************
\name{任笙}
%**********************************联系信息****************************************
%第一个括号里写手机号,第二个写邮箱
\contactInfo{16605242665}{rensheng0410@outlook.com}
%**********************************其他信息****************************************
%在大括号内填写其他信息,最多填写4个,但是如果选择不填信息,
%那么大括号必须空着不写,而不能删除大括号。
%\otherInfo后面的四个大括号里的所有信息都会在一行输出
%如果想要写两行,那就用两次这个指令(\otherInfo{}{}{}{})即可
\otherInfo{性别:男}{籍贯:江苏}{}{}
% 求职意向
\otherInfo{求职意向:软件开发\;人工智能}{}{}{}
\otherInfo{\faGithubAlt Github: \href{https://github.com/ashengstd}{https://github.com/ashengstd}}{}{}{}
%*********************************照片**********************************************
%照片需要放到images文件夹下,名字必须是you.jpg,如果不需要照片可以不添加此行命令
%0.15的意思是,照片的宽度是页面宽度的0.15倍,调整大小,避免遮挡文字
\yourphoto{0.12}
%**********************************正文**********************************************


%***大标题,下面有横线做分割
%***一般的标题有:教育背景,实习(项目)经历,工作经历,自我评价,求职意向,等等
\section{教育背景}


%***********一行子标题**************
%***第一个大括号里的内容向左对齐,第二个大括号里的内容向右对齐
%***\textbf{}括号里的字是粗体,\textit{}括号里的字是斜体


%***********列举*********************
%***可添加多个\item,得到多个列举项,类似的也可以用\textbf{}、\textit{}做强调


\datedsubsection{\textbf{南京航空航天大学},\textbf{\{211,双一流\}},\textit{本科}}{2021.09 - 2025.07}
\begin{itemize} [parsep=1ex]
  \item \textbf{人工智能}
  \item \textbf{GPA 3.5/5.0}
  \item \textbf{Python程序设计语言}:99 \textbf{C++ 程序设计}:93 \textbf{最优化方法}:78 \textbf{模式识别实验}:99 \textbf{自然语言处理}:91 \textbf{操作系统}:90 \textbf{人工智能综合课程设计}:93
\end{itemize}

\section{技术栈}
\begin{itemize}[parsep=0.5ex]
    \item Python, Pytorch
    \item Linux/Shell
    \item Git
    \item Flutter
  \end{itemize}


\section{获奖经历}
\begin{itemize}
    \item 大一,大三学年优秀奖学金\{前20\%\}
    \item CET-4 550\quad CET-6 488分
    \item 2024 美国大学生数学建模比赛\{Sucesseful Mention\}
\end{itemize}

\section{项目经历}

\datedsubsection{\textbf{非机动车逆向检测算法-大创}\{Python\}}{2023.5 - 2024.5}
\begin{itemize}[parsep=0.5ex]
  \item \textbf{国家级立项}
  \item 参与项目的核心开发,主攻计算机视觉方向,结合角度检测算法和\textbf{mmyolo}实现了一个高效的目标检测和方向判断的pipeline。优化了数据处理流程,提升了模型的准确性和计算效率。
\end{itemize}

% \datedsubsection{\textbf{数据挖掘课设}\{Python\}}{2023.10 - 2023.12}
% \begin{itemize}[parsep=0.5ex]
%   \item 结合传统机器学习模型与\textbf{GBDT}(Gradient Boosting Decision Trees),设计并实现了一个针对长短期负类样本筛选的算法。该算法在提高数据预处理效率的同时,也有效提升了后续分类模型的预测准确率。 
% \end{itemize}

\datedsubsection{\textbf{Sqlite-Flutter-APP}\{Flutter\}}{2024.4 - 2024.6}
\begin{itemize}[parsep=0.5ex]
  \item 设计并实现了基于\textbf{Sqlite}和\textbf{Flutter}的移动端APP,成功集成了数据库与前端的无缝交互。实现了增删改查(CRUD)功能,使用户可以高效管理应用内的数据。
\end{itemize}

\datedsubsection{\textbf{MyGPT}\{Pytorch\}}{2024.9 - 2024.11}
\begin{itemize}[parsep=0.5ex]
  \item 基于自定义的\textbf{GPT}模型,设计并实现了一个高效的文本生成和对话模拟系统,能够在多个任务上进行高质量的生成任务。
\end{itemize}

% \datedsubsection{\textbf{YoloV5-Android-APP}\{Java\}}{2024.8 - 2024.9}
% \begin{itemize}[parsep=0.5ex]
%   \item 开发了一个移动端应用,实现了实时剪刀石头布游戏图像识别功能。该功能支持通过摄像头实时检测用户的手势,并进行自动判定。
% \end{itemize}

\datedsubsection{\textbf{云节点的资源调度-企业合作实习项目}\{Pytorch\}}{2024.7 - 2024.11}
\begin{itemize}[parsep=0.5ex]
  \item 基于\textbf{DROO}神经网络与线性规划的组合优化算法,设计并实现了一个高效的云服务资源调度系统。负责整个项目的核心算法实现,包括模型训练与调优,最终成功提升了资源调度的效率和系统的稳定性。 
\end{itemize}

\datedsubsection{\textbf{llm\_eval-大预言模型的推理性能测试CLI}\{Python\}}{2024.11 - 2025.1}
\begin{itemize}[parsep=0.5ex]
  \item 基于\textbf{LiteLLM}实现了一个CLI工具,用于测试大语言模型的推理性能,用\textbf{Github Action}实现了开发、测试、编译、分发的整套CI/CD,实现了对GSM、GSM-Symbolic、SimpleBench等多个数据集的支持,通过LiteLLM的API,实现了任意Provider的模型API调用,同时基于\textbf{mkdocs}实现了一个简单的\href{http://ashengstd.github.io/llm_evaluation_in_reasoning/}{Documentations}。
\end{itemize}

\section{自我评价}
\quad\quad 热爱计算机技术,热爱学习,热爱开源精神,大学期间积极参加开源社区举办的活动和讲座。对新技术和新的计算机分支学科保有兴趣,关注计算机的相关科技知识和新闻。

\end{document}
